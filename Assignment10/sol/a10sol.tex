\documentclass{article}

\usepackage{fancyhdr}
\usepackage{extramarks}
\usepackage{amsmath,siunitx}
\usepackage{amsthm}
\usepackage{bm}
\usepackage{amssymb}
\usepackage{amsfonts}
\usepackage{multirow}
\usepackage{tikz}
\usepackage[plain]{algorithm}
\usepackage{algpseudocode}
\usepackage{changepage}

\usetikzlibrary{automata,positioning}


%
% Basic Document Settings
%

\topmargin=-0.45in
\evensidemargin=0in
\oddsidemargin=0in
\textwidth=6.5in
\textheight=9.0in
\headsep=0.25in

\linespread{1.1}

\pagestyle{fancy}
\lhead{\hmwkTeam}
\chead{\hmwkClass: \hmwkTitle}
\rhead{\firstxmark}
\lfoot{\lastxmark}
\cfoot{\thepage}

\renewcommand\headrulewidth{0.4pt}
\renewcommand\footrulewidth{0.4pt}

\setlength\parindent{0pt}

\newcommand{\setsep}{,    \ }

%
% Create Problem Sections
%

\newcommand{\enterProblemHeader}[1]{
    \nobreak\extramarks{}{Problem \hmwkNumber.\arabic{#1} continued on next page\ldots}\nobreak{}
    \nobreak\extramarks{Problem \hmwkNumber.\arabic{#1} (continued)}{Problem \hmwkNumber.\arabic{#1} continued on next page\ldots}\nobreak{}
}

\newcommand{\exitProblemHeader}[1]{
    \nobreak\extramarks{Problem \hmwkNumber.\arabic{#1} (continued)}{Problem \hmwkNumber.\arabic{#1} continued on next page\ldots}\nobreak{}
    \stepcounter{#1}
    \nobreak\extramarks{Problem \hmwkNumber.\arabic{#1}}{}\nobreak{}
}

\setcounter{secnumdepth}{0}
\newcounter{partCounter}
\newcounter{homeworkProblemCounter}
\setcounter{homeworkProblemCounter}{1}
\nobreak\extramarks{Problem \arabic{homeworkProblemCounter}}{}\nobreak{}

%
% Homework Problem Environment
%
% This environment takes an optional argument. When given, it will adjust the
% problem counter. This is useful for when the problems given for your
% assignment aren't sequential. See the last 3 problems of this template for an
% example.
%
\newenvironment{homeworkProblem}[2][-2]{
    \ifnum#1>0
        \setcounter{homeworkProblemCounter}{#1}
    \fi
    \section{Problem \hmwkNumber.\arabic{homeworkProblemCounter} #2}
    \setcounter{partCounter}{1}
    \enterProblemHeader{homeworkProblemCounter}
}{
    \exitProblemHeader{homeworkProblemCounter}
}

%
% Homework Details
%   - Title
%   - Due date
%   - Class
%   - Section/Time
%   - Instructor
%   - Author
%
\newcommand{\hmwkNumber}{10}
\newcommand{\hmwkTitle}{Exercise Sheet \hmwkNumber}
\newcommand{\hmwkClass}{NNTI}
\newcommand{\hmwkTeam}{Team \#25}
\newcommand{\hmwkAuthorName}{\hmwkTeam: \\ Camilo Martínez 7057573, cama00005@stud.uni-saarland.de \\ Honglu Ma 7055053, homa00001@stud.uni-saarland.de}

%
% Title Page
%

\title{
    % \vspace{2in}
    \textmd{\textbf{\hmwkClass:\ \hmwkTitle}}\\
}

\author{\hmwkAuthorName}
\date \today

\renewcommand{\part}[1]{\textbf{\large Part \Alph{partCounter}}\stepcounter{partCounter}\\}

%
% Various Helper Commands
%

% Useful for algorithms
\newcommand{\alg}[1]{\textsc{\bfseries \footnotesize #1}}

% For derivatives
\newcommand{\deriv}[1]{\frac{\mathrm{d}}{\mathrm{d}x} (#1)}

% For partial derivatives
\newcommand{\pderiv}[2]{\frac{\partial}{\partial #1} (#2)}

% Integral dx
\newcommand{\dx}{\mathrm{d}x}

% Alias for the Solution section header
\newcommand{\solution}{\textbf{\large Solution}}

% Probability commands: Expectation, Variance, Covariance, Bias
\newcommand{\E}{\mathrm{E}}
\newcommand{\Var}{\mathrm{Var}}
\newcommand{\Cov}{\mathrm{Cov}}
\newcommand{\Bias}{\mathrm{Bias}}

\newenvironment{homeworkSubsection}{\begin{adjustwidth}{2.5em}{0pt}}{\end{adjustwidth}}

\begin{document}

\maketitle

\begin{homeworkProblem}{Initialization}
    \subsection*{(a)}
    \begin{homeworkSubsection}
        An initialization of zero for all weights will result in a zero output in each layer.
        One can calculate that the gradients for each weight are also zero thus the network is not improving.
    \end{homeworkSubsection}
    \subsection*{(b)}
    \begin{homeworkSubsection}
    For networks with ReLU, a small constant is usually used to initialize bias so that ReLU will not output zero 
    and the gradient will not be zero as well.
    \end{homeworkSubsection}
\end{homeworkProblem}

\begin{homeworkProblem}{Convolutional Neural Networks}
    \subsection*{(a)}
    \begin{homeworkSubsection}
        CNNs are well suited for images because the output after each convolution can represent specific features of the image.
        Instead of reducing the input size by preprocessing the image (the input size is huge) into features (the size is relatively small)
        before feed it into a feed forward neural network (usually we have to decide which feature to extract),
        we use kernels to extract features on the go and let the network to learn/decide how the kernels look like (which features they give).
    \end{homeworkSubsection}

    \subsection*{(b)}
    \begin{homeworkSubsection}
        One can represent such convolution in matrix multiplication form as:
        \[
            \begin{pmatrix}
                w_1 & w_2 & 0 & w_3 & w_4 & 0 & 0 & 0 & 0\\
                0 & w_1 & w_2 & 0 & w_3 & w_4 & 0 & 0 & 0\\
                0 & 0 & 0 & w_1 & w_2 & 0 & w_3 & w_4 & 0\\
                0 & 0 & 0 & 0 & w_1 & w_2 & 0 & w_3 & w_4 \\
            \end{pmatrix}
            \cdot
            \begin{pmatrix}
                x_1\\
                x_2\\
                x_3\\
                x_4\\
                x_5\\
                x_6\\
                x_7\\
                x_8\\
                x_9\\
            \end{pmatrix}
        \]
    \end{homeworkSubsection}
    
    \subsection*{(c)}
    \begin{homeworkSubsection}
        \textbf{Proof by Induction}\\
        \textbf{Base Step: $l = 1$}\\
        Both formulas give the equality: $R_1 = R_0 + (k - 1)s_1$\\
        \textbf{Induction Step: $l = n$}\\
        Now assume that $R_n = 1 + (k - 1) \sum_{j=1}^{n-1}\prod_{i=1}^{j}s_i$.
        And we want to prove that it holds for the case $R_{n+1}$.
        \begin{align*}
            R_{n+1} &= R_n + (k - 1)\prod_{i=1}^{n}s_i\\
                    &= 1 + (k - 1) \sum_{j=1}^{n-1}\prod_{i=1}^{j}s_i + (k - 1)\prod_{i=1}^{n}s_i\\
                    &= 1 + (k - 1) \sum_{j=1}^{n}\prod_{i=1}^{j}s_i\\
        \end{align*}
    \end{homeworkSubsection}
    
    \subsection*{(d)}
    \begin{homeworkSubsection}
        The output shape is 28x28x30 as you have 10 5x5 kernels, 
        each of them will give you 28x28x3 output and after stacking the 10 result together depthwise, 
        we get 28x28x30.
        The number of parameters is 2500 as we have 10 5x5 kernels.
    \end{homeworkSubsection}
\end{homeworkProblem}

\begin{homeworkProblem}{Residual Networks}
    \subsection*{(a)}
    \begin{homeworkSubsection}
        Very deep neural network may encounter exploding/vanishing gradient.
        It is also shown that the training accuracy may drop.
    \end{homeworkSubsection}
\end{homeworkProblem}

\end{document}