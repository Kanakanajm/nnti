\documentclass{article}

\usepackage{fancyhdr}
\usepackage{extramarks}
\usepackage{amsmath}
\usepackage{amsthm}
\usepackage{bm}
\usepackage{amssymb}
\usepackage{amsfonts}
\usepackage{multirow}
\usepackage{tikz}
\usepackage[plain]{algorithm}
\usepackage{algpseudocode}

\usetikzlibrary{automata,positioning}


%
% Basic Document Settings
%

\topmargin=-0.45in
\evensidemargin=0in
\oddsidemargin=0in
\textwidth=6.5in
\textheight=9.0in
\headsep=0.25in

\linespread{1.1}

\pagestyle{fancy}
\lhead{\hmwkTeam}
\chead{\hmwkClass: \hmwkTitle}
\rhead{\firstxmark}
\lfoot{\lastxmark}
\cfoot{\thepage}

\renewcommand\headrulewidth{0.4pt}
\renewcommand\footrulewidth{0.4pt}

\setlength\parindent{0pt}

\newcommand{\setsep}{,    \ }

%
% Create Problem Sections
%

\newcommand{\enterProblemHeader}[1]{
    \nobreak\extramarks{}{Problem \hmwkNumber.\arabic{#1} continued on next page\ldots}\nobreak{}
    \nobreak\extramarks{Problem \hmwkNumber.\arabic{#1} (continued)}{Problem \hmwkNumber.\arabic{#1} continued on next page\ldots}\nobreak{}
}

\newcommand{\exitProblemHeader}[1]{
    \nobreak\extramarks{Problem \hmwkNumber.\arabic{#1} (continued)}{Problem \hmwkNumber.\arabic{#1} continued on next page\ldots}\nobreak{}
    \stepcounter{#1}
    \nobreak\extramarks{Problem \hmwkNumber.\arabic{#1}}{}\nobreak{}
}

\setcounter{secnumdepth}{0}
\newcounter{partCounter}
\newcounter{homeworkProblemCounter}
\setcounter{homeworkProblemCounter}{1}
\nobreak\extramarks{Problem \arabic{homeworkProblemCounter}}{}\nobreak{}

%
% Homework Problem Environment
%
% This environment takes an optional argument. When given, it will adjust the
% problem counter. This is useful for when the problems given for your
% assignment aren't sequential. See the last 3 problems of this template for an
% example.
%
\newenvironment{homeworkProblem}[2][-2]{
    \ifnum#1>0
        \setcounter{homeworkProblemCounter}{#1}
    \fi
    \section{Problem \hmwkNumber.\arabic{homeworkProblemCounter} #2}
    \setcounter{partCounter}{1}
    \enterProblemHeader{homeworkProblemCounter}
}{
    \exitProblemHeader{homeworkProblemCounter}
}

%
% Homework Details
%   - Title
%   - Due date
%   - Class
%   - Section/Time
%   - Instructor
%   - Author
%
\newcommand{\hmwkNumber}{4}
\newcommand{\hmwkTitle}{Exercise Sheet \hmwkNumber}
\newcommand{\hmwkClass}{NNTI}
\newcommand{\hmwkTeam}{Team \#25}
\newcommand{\hmwkAuthorName}{\hmwkTeam: Camilo Martínez 7057573, Honglu Ma 7055053}

%
% Title Page
%

\title{
    % \vspace{2in}
    \textmd{\textbf{\hmwkClass:\ \hmwkTitle}}\\
}

\author{\hmwkAuthorName}
\date \today

\renewcommand{\part}[1]{\textbf{\large Part \Alph{partCounter}}\stepcounter{partCounter}\\}

%
% Various Helper Commands
%

% Useful for algorithms
\newcommand{\alg}[1]{\textsc{\bfseries \footnotesize #1}}

% For derivatives
\newcommand{\deriv}[1]{\frac{\mathrm{d}}{\mathrm{d}x} (#1)}

% For partial derivatives
\newcommand{\pderiv}[2]{\frac{\partial}{\partial #1} (#2)}

% Integral dx
\newcommand{\dx}{\mathrm{d}x}

% Alias for the Solution section header
\newcommand{\solution}{\textbf{\large Solution}}

% Probability commands: Expectation, Variance, Covariance, Bias
\newcommand{\E}{\mathrm{E}}
\newcommand{\Var}{\mathrm{Var}}
\newcommand{\Cov}{\mathrm{Cov}}
\newcommand{\Bias}{\mathrm{Bias}}


\begin{document}

\maketitle
\begin{homeworkProblem}{Bias and Variance}

\subsection*{a)}
Bias indicates how well the model predicts and Variance is a measurement of how similar the models trained with different training set behaves. A complex model results in a low bias and a high variance and a simple model results in a high bias and a low variance.

\subsection*{b)}
Overfitting means low bias but high variance.
Underfitting means high bias but low variance.

\subsection*{c)}
\newcommand{\fh}{\hat{f}}
\begin{align*}
MSE(y, \fh) &= E[(y - \fh(x_0))^2)]\\
&= E[y^2 - 2y\fh+\fh^2]\\
&=E[y^2] - 2E[y\fh] + E[\fh^2]\\
&=E[(f+\varepsilon)^2] - 2E[(f+\varepsilon)\fh] + E[\fh^2]\\
&=E[f^2]+ 2E[f]E[\varepsilon] + E[\varepsilon^2] - 2(E[f\fh]+ E[\varepsilon]E[\fh]) + E[\fh^2]\\
&=f^2+2f\cdot0 + Var(\varepsilon) - 2E[f]E[\fh] - 2\cdot0\cdot E[\fh] + E[\fh^2]\\
&=f^2+Var(\varepsilon) - 2fE[\fh] + E[\fh^2]\\
&=f^2+Var(\varepsilon) - 2fE[\fh] + E[\fh^2] + E[\fh]^2 - E[\fh]^2\\
&=(f^2 - 2fE[\fh] + E[\fh]^2) + (E[\fh^2] - E[\fh]^2)+Var(\varepsilon)\\
&=(f-E[\fh])^2+ (E[\fh^2] - E[\fh]^2)+Var(\varepsilon)\\
&=Bias(\fh)^2+ Var(\fh) +Var(\varepsilon)\\
\end{align*}

\subsection*{d)}
When the training set size goes up, the variance goes down but the bias stays the same
but when the training set is really small, increasing the training set size can increase the bias as well.

\end{homeworkProblem}

\end{document}