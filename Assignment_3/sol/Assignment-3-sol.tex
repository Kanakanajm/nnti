\documentclass{article}

\usepackage{fancyhdr}
\usepackage{extramarks}
\usepackage{amsmath}
\usepackage{amsthm}
\usepackage{bm}
\usepackage{amssymb}
\usepackage{amsfonts}
\usepackage{multirow}
\usepackage{tikz}
\usepackage[plain]{algorithm}
\usepackage{algpseudocode}

\usetikzlibrary{automata,positioning}


%
% Basic Document Settings
%

\topmargin=-0.45in
\evensidemargin=0in
\oddsidemargin=0in
\textwidth=6.5in
\textheight=9.0in
\headsep=0.25in

\linespread{1.1}

\pagestyle{fancy}
\lhead{\hmwkTeam}
\chead{\hmwkClass: \hmwkTitle}
\rhead{\firstxmark}
\lfoot{\lastxmark}
\cfoot{\thepage}

\renewcommand\headrulewidth{0.4pt}
\renewcommand\footrulewidth{0.4pt}

\setlength\parindent{0pt}

\newcommand{\setsep}{,    \ }

%
% Create Problem Sections
%

\newcommand{\enterProblemHeader}[1]{
    \nobreak\extramarks{}{Problem \hmwkNumber.\arabic{#1} continued on next page\ldots}\nobreak{}
    \nobreak\extramarks{Problem \hmwkNumber.\arabic{#1} (continued)}{Problem \hmwkNumber.\arabic{#1} continued on next page\ldots}\nobreak{}
}

\newcommand{\exitProblemHeader}[1]{
    \nobreak\extramarks{Problem \hmwkNumber.\arabic{#1} (continued)}{Problem \hmwkNumber.\arabic{#1} continued on next page\ldots}\nobreak{}
    \stepcounter{#1}
    \nobreak\extramarks{Problem \hmwkNumber.\arabic{#1}}{}\nobreak{}
}

\setcounter{secnumdepth}{0}
\newcounter{partCounter}
\newcounter{homeworkProblemCounter}
\setcounter{homeworkProblemCounter}{1}
\nobreak\extramarks{Problem \arabic{homeworkProblemCounter}}{}\nobreak{}

%
% Homework Problem Environment
%
% This environment takes an optional argument. When given, it will adjust the
% problem counter. This is useful for when the problems given for your
% assignment aren't sequential. See the last 3 problems of this template for an
% example.
%
\newenvironment{homeworkProblem}[2][-2]{
    \ifnum#1>0
        \setcounter{homeworkProblemCounter}{#1}
    \fi
    \section{Problem \hmwkNumber.\arabic{homeworkProblemCounter} #2}
    \setcounter{partCounter}{1}
    \enterProblemHeader{homeworkProblemCounter}
}{
    \exitProblemHeader{homeworkProblemCounter}
}

%
% Homework Details
%   - Title
%   - Due date
%   - Class
%   - Section/Time
%   - Instructor
%   - Author
%
\newcommand{\hmwkNumber}{3}
\newcommand{\hmwkTitle}{Exercise Sheet \hmwkNumber}
\newcommand{\hmwkClass}{NNTI}
\newcommand{\hmwkTeam}{Team \#25}
\newcommand{\hmwkAuthorName}{\hmwkTeam: Camilo Martínez 7057573, Honglu Ma 7055053}

%
% Title Page
%

\title{
    % \vspace{2in}
    \textmd{\textbf{\hmwkClass:\ \hmwkTitle}}\\
}

\author{\hmwkAuthorName}
\date \today

\renewcommand{\part}[1]{\textbf{\large Part \Alph{partCounter}}\stepcounter{partCounter}\\}

%
% Various Helper Commands
%

% Useful for algorithms
\newcommand{\alg}[1]{\textsc{\bfseries \footnotesize #1}}

% For derivatives
\newcommand{\deriv}[1]{\frac{\mathrm{d}}{\mathrm{d}x} (#1)}

% For partial derivatives
\newcommand{\pderiv}[2]{\frac{\partial}{\partial #1} (#2)}

% Integral dx
\newcommand{\dx}{\mathrm{d}x}

% Alias for the Solution section header
\newcommand{\solution}{\textbf{\large Solution}}

% Probability commands: Expectation, Variance, Covariance, Bias
\newcommand{\E}{\mathrm{E}}
\newcommand{\Var}{\mathrm{Var}}
\newcommand{\Cov}{\mathrm{Cov}}
\newcommand{\Bias}{\mathrm{Bias}}


\begin{document}

\maketitle

\begin{homeworkProblem}{- Linear Regression}

    \subsection*{(a)}
Plug the two points in the equation, we get:
\begin{align*} 
2w + b &= 4 \\ 
4w + b &=  5
\end{align*}
Solve the equations, we have:
\begin{align*} 
w &= \frac{1}{2}\\
b &= 3
\end{align*}
    \subsection*{(b)}
It does not lie on the line above. \\
Proof by contradiction:\\
Assume that the point lies on the line, it must satisfy the equation $Y = \frac{1}{2}X+3$.\\
Plug the point in the equation, we get: $\frac{9}{2} = 6$ which is false.
    \subsection*{(c, short)}
    By using the result we get from Exercise 2d in Assignment 2 $$f(x) = ||Bx - c||_2^2, \nabla_xf(x) = 2B^\top(Bx - c)$$ we get:
    $$\nabla_w\mathrm{MSE}_{train} = \nabla_w\frac{1}{m}||X_{train}w - y_{train}||_2^2 = \frac{1}{m}(2X_{train}^\top(X_{train}w - y_{train})) = X_{train}^\top X_{train}w - X_{train}^\top y_{train} = 0$$\\ which can be written as\\
    $$X_{train}^\top X_{train}w = X_{train}^\top y_{train}$$\\
    so\\
    $$w = (X_{train}^\top X_{train})^{-1} X_{train}^\top y_{train}$$\\
    
    \subsection*{(c)}
    The Mean Squared Error (MSE) equation is given by:
    
    \[
        \mathrm{MSE}_{train} = \frac{1}{m}\sum_{i=1}^m {(\hat{y}_{train}^{(i)} - y_{train}^{(i)})^2}
    \]
    
    for a linear regression model \(Y = wX + b\), where \(\hat{y}_{train}^{(i)}\) is the predicted value, \(y_{train}^{(i)}\) is the actual value, and \(m\) is the number of training examples.\\
    
    To minimize this equation, we need to find the value of \(w\) where the gradient of MSE with respect to \(w\) is equal to zero. For that, we calculate the derivative of MSE with respect to \(w\) and set it to zero:
    
    \begin{equation}\label{first}
        \frac{\partial \mathrm{MSE}_{train}}{\partial w} = 0 \rightarrow 
        \frac{\partial}{\partial w} 
        \begin{bmatrix}
            \frac{1}{m}\sum_{i=1}^m {(\hat{y}_{train}^{(i)} - y_{train}^{(i)})^2}
        \end{bmatrix} = 0
    \end{equation}
    
    From the linear regression model equation, we know that \(\hat{y}_{train}^{(i)} = wX^{(i)} + b\). We can plug this into (\ref{first}):

    \begin{equation}\label{first}
        \begin{split}
            \frac{\partial}{\partial w} 
            \begin{bmatrix}
                \frac{1}{m}\sum_{i=1}^m {(wX^{(i)} + b - y_{train}^{(i)})^2}
            \end{bmatrix} = \frac{2}{m} \sum_{i=1}^m (\hat{y}_{train}^{(i)} - y_{train}^{(i)})X^{(i)}
        \end{split}
    \end{equation}

    
    
    3. **Setting the Derivative to Zero:**
    \[
    \frac{2}{m} \sum_{i=1}^m (\hat{y}_{train}^{(i)} - y_{train}^{(i)})X^{(i)} = 0
    \]
    
    4. **Simplify the Equation:**
    \[
    \sum_{i=1}^m (\hat{y}_{train}^{(i)} - y_{train}^{(i)})X^{(i)} = 0
    \]
    
    5. **Expand the Summation:**
    \[
    \sum_{i=1}^m \hat{y}_{train}^{(i)}X^{(i)} - \sum_{i=1}^m y_{train}^{(i)}X^{(i)} = 0
    \]
    
    6. **Move the Second Term to the Other Side:**
    \[
    \sum_{i=1}^m \hat{y}_{train}^{(i)}X^{(i)} = \sum_{i=1}^m y_{train}^{(i)}X^{(i)}
    \]
    
    7. **Substitute \(\hat{y}_{train}^{(i)} = wX^{(i)} + b\):**
    \[
    \sum_{i=1}^m (wX^{(i)} + b)X^{(i)} = \sum_{i=1}^m y_{train}^{(i)}X^{(i)}
    \]
    
    8. **Expand the First Term:**
    \[
    \sum_{i=1}^m w(X^{(i)})^2 + bX^{(i)} = \sum_{i=1}^m y_{train}^{(i)}X^{(i)}
    \]
    
    9. **Move the Second Term to the Other Side:**
    \[
    \sum_{i=1}^m w(X^{(i)})^2 = \sum_{i=1}^m y_{train}^{(i)}X^{(i)} - bX^{(i)}
    \]
    
    10. **Divide by the Sum of Squares of \(X^{(i)}\):**
    \[
    w = \frac{\sum_{i=1}^m y_{train}^{(i)}X^{(i)} - bX^{(i)}}{\sum_{i=1}^m (X^{(i)})^2}
    \]
    
    This is the expression for \(w\) that minimizes the MSE.
    \subsection*{(d)}
    The difference between the line and x value can be seen as $f^{-1}(y_{train}) - X_{train}$ where we use the intercept $y$ to get the feature $X$ which is not what we want to achieve.
\end{homeworkProblem}

\begin{homeworkProblem}{ - PCA as Autoencoder}
\subsection*{(a)}
We minimize the reconstruction error by solving $\nabla_c||x - Dc||_2^2  = 0$:
\begin{alignat*}{2}
                    & & \nabla_c\,||x - Dc||_2^2  & = 0\\
   \Rightarrow\quad & & \nabla_c\,(x - Dc)^\top (x - Dc) & = 0\\
   \Rightarrow\quad & & \nabla_c\,(x^\top - (Dc)^\top)(x - Dc)& = 0\\
   \Rightarrow\quad & & \nabla_c\,x^\top x - x^\top Dc - (Dc)^\top x + (Dc)^\top (Dc)& = 0\tag{1}\\
   \Rightarrow\quad & & \nabla_c\,x^\top x - x^\top Dc - ((Dc)^\top x)^\top + (Dc)^\top (Dc)& = 0\tag{2}\\
   \Rightarrow\quad & & \nabla_c\,x^\top x - x^\top Dc - x^\top Dc + (Dc)^\top (Dc)& = 0\\
   \Rightarrow\quad & & \nabla_c\,x^\top x - 2x^\top Dc + c^\top(D^\top D)c& = 0\\
   \Rightarrow\quad & & \nabla_c\,x^\top x - 2x^\top Dc + c^\top I_l c& = 0\\
   \Rightarrow\quad & & \nabla_c\,x^\top x - 2x^\top Dc + c^\top c& = 0\tag{3}\\
   \Rightarrow\quad & & 0 - 2 D^\top x + 2c & = 0\tag{4}\\
   \Rightarrow\quad & & c & = D^\top x \\
\end{alignat*}
Clarification of Step 0 to 1: because $(Dc)^\top x$ is a scalar so $(Dc)^\top x = ((Dc)^\top x)^\top$; clarification of Step 2 to 3: we calculate the gradient with respect to $c$; the first term $\nabla_c\,x^\top x = 0$ because it does not have $c$ term, it acts as a constant; the second term can be rewritten as $\nabla_c\, -2(D^\top x)^\top c$ so that we can apply the result of Exercise 2a in Assignment 2, we get $\nabla_c\, -2(D^\top x)^\top c = D^\top x$; the third term can be rewritten as $\nabla_c c^\top I_l c$ and by the result of Exercise 2b in Assignment 2 we know that $\nabla_c c^\top I_l c = I_l c + I_l^\top c = 2c$

\subsection*{(b)}

\end{homeworkProblem}

\begin{homeworkProblem}{- PCA}
    See attached .ipynb solution in .zip file.

\end{homeworkProblem}

\end{document}